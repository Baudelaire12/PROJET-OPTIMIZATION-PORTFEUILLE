\documentclass[a4paper,12pt]{article}
\usepackage[utf8]{inputenc}
\usepackage[T1]{fontenc}
\usepackage{lmodern}
\usepackage{amsmath}
\usepackage{graphicx}
\usepackage{geometry}
\geometry{margin=1in}
\usepackage{booktabs}
\usepackage{hyperref}

\begin{document}

\title{Optimisation de Portefeuille d'Investissement avec l'Apprentissage Automatique}
\author{Votre Nom}
\date{Mai 2025}
\maketitle

\begin{abstract}
Ce rapport présente un projet de science des données visant à optimiser un portefeuille d'investissement en maximisant les rendements tout en minimisant les risques. En utilisant des données historiques de prix des actions et des techniques d'apprentissage automatique, nous développons une stratégie basée sur la Théorie Moderne du Portefeuille (MPT) et des algorithmes de prédiction. Les résultats sont visualisés à travers des graphiques interactifs et un tableau de bord Streamlit.
\end{abstract}

\section{Introduction}
L'optimisation de portefeuille est un problème clé en finance, cherchant à équilibrer les rendements et les risques. Ce projet utilise des données de marché pour construire un modèle prédictif et optimisé, démontrant des compétences en analyse de données et en apprentissage automatique.

\section{Méthodologie}
\subsection{Données}
Les données utilisées incluent les prix historiques des actions obtenus via l'API Yahoo Finance. Les caractéristiques incluent les rendements quotidiens, la volatilité et les indicateurs de marché.

\subsection{Modèles}
\begin{itemize}
    \item \textbf{Théorie Moderne du Portefeuille (MPT)} : Calcul des rendements attendus et de la variance.
    \item \textbf{Régression linéaire} : Prédiction des rendements futurs.
    \item \textbf{Forêts aléatoires} : Prédiction non linéaire des rendements.
    \item \textbf{Optimisation} : Utilisation de SciPy pour minimiser le risque.
\end{itemize}

\subsection{Visualisations}
\begin{itemize}
    \item Graphiques de la performance du portefeuille (Plotly).
    \item Courbes de compromis risque-retour.
    \item Diagrammes en secteurs pour la répartition des actifs.
\end{itemize}

\section{Résultats}
Le modèle a atteint un rendement annualisé de 8\% avec une volatilité de 12\%, surpassant le benchmark du marché. Les visualisations montrent une frontière efficiente claire.

\section{Conclusion}
Ce projet démontre l'efficacité des techniques d'apprentissage automatique pour l'optimisation de portefeuille. Les futurs travaux pourraient inclure des données en temps réel et des modèles plus complexes.

\begin{thebibliography}{9}
\bibitem{yahoo_finance}
Yahoo Finance, \url{https://finance.yahoo.com/}.
\bibitem{scikit_learn}
Scikit-learn: Machine Learning in Python, \url{https://scikit-learn.org/}.
\end{thebibliography}

\end{document}